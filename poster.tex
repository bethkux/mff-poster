
% POSTER EXAMPLE
%
% This is an example of a relatively sane poster. The box structure (and the
% narrative in general) is what I would expect, but it is completely
% non-mandatory; you may include whatever you want. Preferably, erase the
% existing box structure after you read it, and start from scratch.
%
% The main communication requirements for the poster that should be satisfied
% are as such:
%
% - At the defense, it should help you talk for around 10 minutes about your
%   thesis, and convince the committee that you did something interesting and
%   sufficiently complicated. Prepare pictures that explain your main results.
%
% - It should quickly communicate the main idea of your thesis to a random
%   educated by-walker. Ideally, a moderately-witted MFF graduate who has never
%   heard about your thesis before should be able to get the main "rough idea"
%   in less than 1 minute by just looking at the poster.

% modify the fontscale parameter to make everything slighly bigger or smaller.
\documentclass[portrait,a0paper,fontscale=0.33]{baposter}

\usepackage[utf8]{inputenc}
\usepackage[T1]{fontenc}

% FONT CHOICES
% Posters do not need to be PDF/A; you can choose any relatable font from the
% TeX font catalogue without much risk. Sans-serif fonts are suggested for the
% posters; see https://tug.org/FontCatalogue/sansseriffonts.html
\usepackage[sfdefault]{Fira Sans}
\usepackage{hyperref} 
%\usepackage[default]{droidsans}
%\usepackage[math]{iwona}
%\usepackage[defaultfam]{montserrat}
%\usepackage{cmbright}
%\usepackage{yfonts}\renewcommand{\familydefault}{\frakdefault}
\usepackage{wrapfig}
\usepackage{color}
\usepackage{graphicx}
\usepackage{multicol}
\usepackage{tasks}
\usepackage{wrapfig}
\usepackage{comment}
\usepackage{amssymb,amsmath}
\usepackage[export]{adjustbox} %allows using valign with \includegraphics

\renewcommand{\arraystretch}{1.5}

\usetikzlibrary{positioning}

% A WORD ABOUT COLORS
%
% This template is prepared with a relatively neutral gray background that
% gives decent box borders (with white and darker gray), does not clash with
% many colors (except for violet-brown and other mushroomish colors, perhaps)
% and gives a lot of space for highlighting stuff.
%
% Generally, other color variations are good too; there are no strict rules on
% the colors. Good choices include:
%
% - white backgrounds and differentiation of box headers by color (see
%   headerFontColor)
%
% - various slightly tinted backgrounds (try red!10 instead of black!3)
% 
% - dark backgrounds
%
% Keep in mind:
% - The normal "informative" text and figures should be DARK on LIGHT
%   background, not the other way around.
%
% - If you want a dark background, soften (darken) the box backgrounds a bit so
%   that they do not "shine" too much from the poster. Use \color{white} for
%   the heading, and switch the UK/MFF logos to white (see contents of logos/).
%
% - Do not mix too many color hues together. Most hues have their widely
%   accepted meaning (green: good result, red: problem, blue: information,
%   yellow: highlighter, brown: serious problem, violet: something really
%   weird/interesting/magic, depending on the shade).

\begin{document}

\color{black!80} % default font color
\begin{poster}{grid=false,
	eyecatcher=true,
	background=plain,
	bgColorOne=black!3, % background color
	columns=2,
	headerborder=none,
	textborder=none,
	headershape=rectangle,
	headershade=plain,
	boxshade=plain,
	boxColorOne=white,
	headershade=plain,
	headerColorOne=black!15, % box header background color
	headerFontColor=black,
	}%
	{\includegraphics[height=6em]{logos/mff-black.pdf}}
	{TaleCraft — Framework for 2D\\ Point-and-Click Adventure Games}
	{\vspace{1ex} Alžbeta Kulichová | Supervisor: Mgr. Pavel Ježek, Ph.D.}
	{\includegraphics[height=6em]{logos/uk-red.pdf}}


%
% LEFT COLUMN
%

\begin{posterbox}[column=0, span=1, name=background]{Introduction}
A typical point-and-click adventure game is a genre that emphasizes puzzle solving and often involves mystery and exploration. Players interact with in-game environments through a mouse cursor that is used to click on objects, locations, or characters in a scene to trigger actions or dialogues. Despite its seemingly straightforward gameplay, creating a point-and-click adventure game involves implementing a variety of systems, including movement, dialogue and more. For inexperienced developers, designing these can be difficult. Furthermore, there is a lack of accessible, beginner-friendly tools dedicated specifically to this genre in Unity Asset Store. To fill this demand, the goal of this thesis is to design and implement an accessible framework tailored to the development of 2D point-and-click adventure games in Unity. This framework, called \textit{TaleCraft}, will prioritize functionality and user-friendliness to help developers create games without need for advanced programming skills.
\end{posterbox}

\begin{posterbox}[column=0, span=1, name=goals, below=background, headerColorOne=cyan!40, boxColorOne=cyan!10]{Goals}
For this thesis we aim to meet these two following goals:
\begin{itemize}
\item \textbf{Framework}. We implement a framework in Unity with the following systems:
\begin{itemize}
    \item Inventory
    \item Commands
    \item Movement
    \item Dialogue
\end{itemize}
\item \textbf{Demo games.} To showcase the utility of our framework, we create two demo games.
\end{itemize}

\end{posterbox}

\begin{posterbox}[column=0, span=1, name=architecture, below=goals]{Architecture}

The framework is divided into five systems, each containing distinct features of TaleCraft. The Command System forms the core of the framework, while the Inventory and Dialogue Systems are optional components that can be configured according to the user's needs.
\begin{center}
\includegraphics[width=0.3\linewidth]{img/framework.png}
\end{center}
\end{posterbox}

\begin{posterbox}[column=0, span=1, name=cs, below=architecture]{Command System}

\begin{wrapfigure}{R}{0.34\textwidth}
\centering
\includegraphics[width=0.3\textwidth]{img/image_2025-07-08_214350710.png}
\end{wrapfigure}

Typical point-and-click games use either a command-based or a context-based system to let the player interact with the world. To enable the use of both of these methods, we implemented a system that lets the user of the framework add new commands. If needed, these commands can be displayed on the screen and made selectable by the player. The user can define the logic behind various actions by setting up conditions. When these conditions are met, the corresponding actions are executed.  The system also offers additional features, such as the ability to show a sentence describing the actions that the player is about to take.


\end{posterbox}

\begin{posterbox}[column=0, span=1, name=ws, below=cs]{Walking System}
Normally, the player has the ability to move around the walking area. TaleCraft supports this functionality by representing walkable areas using polygons. At least one polygon is required to define the main walkable region, while additional optional polygons can be used to define obstacles that characters must navigate around. The image below shows a scene which uses polygons to determine where the characters will walk. Pathfinding is handled by an implementation of the A* algorithm. The system also supports character scaling based on their position, which creates the illusion of depth and makes the 2D environment appear more three-dimensional. 

\begin{center}
\includegraphics[width=0.75\linewidth]{img/walkable_map3.png}
\end{center}
\end{posterbox}

%
% RIGHT COLUMN
%
% It is usually best to fill most of the poster with your results and
% conclusions. Again, use simple annotated pictures wherever possible. Plots
% with measurements are perfect, tables are also good.
%

\begin{posterbox}[column=1, span=1, name=is]{Inventory}
To let the player collect and use items, the framework includes an inventory system. A key feature is the ability to reference inventory items across different scenes. In TaleCraft, these items are \verb|ScriptableObject|s, which makes it easy to create new items in the project and ensures they can be accessed globally. These objects store essential data such as the item's name and description, which can be used for features like displaying tags when the player hovers over an item with the mouse cursor.
\end{posterbox}

\begin{posterbox}[column=1, span=1, name=ds, below=is, %bottomaligned=info
]{Dialogue}
TaleCraft uses graphs for representing graphs. They are ideal when dealing with complex branching conversations. They also make it straightforward to implement cyclic paths, allowing characters to revisit or repeat parts of a conversation. In TaleCraft, the dialogue system is built using Unity’s \textit{GraphView} framework. Several types of nodes are used to construct these graphs: 

\begin{center}
\includegraphics[width=0.65\linewidth]{img/nodes1.png}
\includegraphics[width=0.7\linewidth]{img/nodes2.png}
\end{center}

%{\footnotesize
%\begin{tasks}[label=\textbullet](3)
%       \task Start
%        \task End
%        \task Dialogue
%        \task Choice
%        \task Event
%        \task Branch
%\end{tasks}}
\end{posterbox}


\begin{posterbox}[column=1, name=result2, below=ds]{Result: Demo}
\begin{center}
\includegraphics[width=0.506\linewidth]{img/manual.png}
\includegraphics[width=0.486\linewidth]{img/manual-tsomi.png}
\end{center}
\end{posterbox}


\begin{posterbox}[column=1, name=conclusion, below=result2, headerColorOne=yellow!80!orange!95!black, boxColorOne=yellow!33]{Conclusion}

%\begin{wrapfigure}{r}{0.4\textwidth}
%\includegraphics[width=1\linewidth]{img/qr.png} 
%\end{wrapfigure}

The resulting framework delivers on its objectives and serves as a foundation for developing 2D point-and-click adventure games. In addition, two simple demo games that demonstrate the capabilities of our framework have been developed. Suggestions for future work include the implementation of features that should eventually be part of the framework, the most important being animation, sound, and saving system. 

\end{posterbox}


\begin{posterbox}[column=1, name=info, below=conclusion]{Info}

\begin{comment}
\begin{wrapfigure}{R}{0.35\textwidth}
\centering
\includegraphics[width=0.16\textwidth]{img/qr.png}
\end{wrapfigure}
\end{comment}
%\begin{tabular}{@{}p{0.65\linewidth} p{0.3\linewidth}@{}}
% Left column: everything in a parbox
 


\vspace{5mm}
    \textbf{Contact:} \\
    \href{mailto:kulichova.alzbeta@gmail.com}{kulichova.alzbeta@gmail.com} \\[1ex]
    \textbf{Project GitHub Page:} \\
    \href{https://github.com/bethkux/TaleCraft}{https://github.com/bethkux/TaleCraft}\\
    

%&
% Right column: QR image, top aligned

%\end{tabular}

%\begin{comment}
\begin{flushright}
\raisebox{10pt}[0pt][0pt]{%
\includegraphics[width=0.22\linewidth]{img/qr.png}%
}
\end{flushright}
%\end{comment}
\end{posterbox}




\end{poster}
\end{document}
